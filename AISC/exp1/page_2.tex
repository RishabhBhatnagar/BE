\documentclass{article}
\usepackage[margin=0.6in]{geometry}
\usepackage{microtype}
\title{\textbf{Family Tree}}
\author{Rishabh Bhatnagar}
\date{July 15}
 
\begin{document}
    \begin{titlepage}
        \maketitle

        \section{CODE:}
        \begin{verbatim}
        %%%%%%%%%%%%%%%%%%%%%%%%%%%%%%%%%%%%%%%%%%%%%%%%%%%%%%%%%%%%%
        male(aditya).
        male(rishabh).

        male(dhruv).
        male(anurag).

        male(sushil).
        male(kshitij).
        male(ritesh).
        male(kharagsingh).
        male(vipin).

        female(ankita).
        female(neha).
        female(khyati).
        female(ritu).
        female(mahima).
        female(abha).
        female(shantidevi).
        female(kantidevi).
        
        %%%%%%%%%%%%%%%%%%%%%%%%%%%%%%%%%%%%%%%%%%%%%%%%%%%%%%%%%%%%%
        wife(ritu, sushil).
        wife(mahima, ritesh).
        wife(abha, kshitij).
        wife(kantidevi, kharagsingh).
        wife(shantidevi, vipin).

        mother(ritu, aditya).
        mother(ritu, rishabh).
        mother(abha, neha).
        mother(mahima, dhruv).
        mother(abha, anurag).
        mother(shantidevi, sushil).
        mother(shantidevi, kshitij).
        mother(kantidevi, ritu).
        mother(kantidevi, ritesh).
        %%%%%%%%%%%%%%%%%%%%%%%%%%%%%%%%%%%%%%%%%%%%%%%%%%%%%%%%%%%%%
        % Rules %
        husband(A, B)       :- wife(B, A), male(A).
        married(A, B)       :- wife(A, B).
        married(A, B)       :- husband(A,B).
        child(A, B)         :- husband(B, C), mother(C, A).
        child(A, B)         :- mother(B, A).
        son(A, B)           :- child(A, B), male(A).
        daughter(A, B)      :- child(A, B), female(A).
        parent(A, B)        :- mother(A, B).
        parent(A, B)        :- father(A, B).
        father(A, B)        :- child(B, A), mother(C, B), wife(C, A).
        brother(A, B)       :- mother(C, A), mother(C, B), male(A).
        sister(A, B)        :- mother(C, A), mother(C, B), female(A).
        grandparent(A, B)   :- parent(C, B), parent(A, C).
        grandfather(A, B)   :- grandparent(A, B), male(A).
        grandmother(A, B)   :- grandparent(A, B), female(A).
        grandson(A, B)      :- granparent(B, A), male(A).
        granddaughter(A, B) :- grandparent(B, A), female(A).
        sibling(A, B)       :- parent(C, A), brother(C, D), child(B, D).
        sibling(A, B)       :- parent(C, A), sister(C, D), child(B, D).
        \end{verbatim}

        \section{\textbf{\textit{OUTPUT}}}
        \begin{verbatim}
            rishabh@rishabh:~/Desktop/BE/AISC/exp1\$ prolog
            GNU Prolog 1.3.0
            By Daniel Diaz
            Copyright (C) 1999-2007 Daniel Diaz
            | ?- [exp1_family_tree].
            compiling /home/rishabh/Desktop/BE/AISC/exp1/exp1_family_tree.pl for byte code...
            /home/rishabh/Desktop/BE/AISC/exp1/exp1_family_tree.pl compiled, 59 lines read 
            - 7776 bytes written, 9 ms

            yes
            | ?- father(sushil, aditya).

            true ? 

            yes
            | ?- son(rishabh, sushil).

            true ? 

            yes
            | ?- brother(rishabh, aditya).

            true ? 

            yes
            | ?- sibling(neha, rishabh).

            true ? 

            yes
            | ?- sister(ritu, abha)
            .

            no
            | ?- grandparent(kharagsingh, rishabh).

            true ? 

            yes
            | ?- married(mahima, ritesh).

            true ? 

            yes
            | ?- male(shantidevi), male(abha)
            .

            no
            | ?- %% inexistent data query %%
            .
            .
            uncaught exception: error(syntax_error('user_input:13 
                    (char:1) expression expected'),read_term/3)
            | ?- male(hacker).

            no
            | ?- mother(mahima, dhruv), father(ritesh, dhruv).

            true ? 

            (4 ms) yes
            | ?- married(mahima, ritesh).

            true ? 

            yes
            | ?- sibling(rishabh, anurag).

            true ? 
        \end{verbatim}
    \end{titlepage}
\end{document}

