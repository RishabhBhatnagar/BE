\documentclass{article}


%%%%%%%%%%%%%%%%%%%%%%%%%%%%%Package Imports%%%%%%%%%%%%%%%%%%%%%%%%%%%%%%%%%%%
\usepackage[margin=2cm]{geometry}
\usepackage{graphicx}
\usepackage{multicol}
\usepackage{url}
\usepackage{float}


%%%%%%%%%%%%%%%%%%%%%%%%%%%%%Initial Settings%%%%%%%%%%%%%%%%%%%%%%%%%%%%%%%%%%
\graphicspath{./}
\setlength{\columnsep}{1cm}


%%%%%%%%%%%%%%%%%%%%%%%%%%%%%Custom Commands%%%%%%%%%%%%%%%%%%%%%%%%%%%%%%%%%%%
\newcommand{\addImage}[5]{
    \begin{figure}[#1]
        \centering
        \includegraphics[width=#2]{#3}
        \caption{#4}
        \label{fig:#5}
    \end{figure}
}

\begin{document}
   \title{Case Study on DSIP and it's application.}                           
    \author{Rishabh Bhatnagar}
    \maketitle
    \tableofcontents
    \newpage

    \begin{multicols}{2}                % two columns

    \section{Signal Definition}
        Anything that carries information can be called as signal. 
        It can also be defined as a physical quantity that varies with time, 
        temperature, pressure or with any independent variables such as 
        speech signal or video signal. The process of operation in which the 
        characteristics of a signal (Amplitude, shape, phase, frequency, etc.) 
        undergoes a change is known as signal processing. 
        \cite{tpSignal}

    \section{Types of Signal}
        \subsection{Analog Signal}
            An analog signal is any continuous signal for which the 
            time-varying feature (variable) of the signal is a representation 
            of some other time varying quantity, i.e., analogous to another 
            time varying signal. For example, in an analog audio signal, the 
            instantaneous voltage of the signal varies continuously with the 
            pressure of the sound waves.
            \cite{wikiAnalog}
        \addImage{H}{6cm}{analogSignal.jpeg}{Analog Signal}{analog signal}
        \subsection{Digital Signal}
            A digital signal refers to an electrical signal that is converted 
            into a pattern of bits. Unlike an analog signal, which is a 
            continuous signal that contains time-varying quantities, a digital 
            signal has a discrete value at each sampling point. The precision 
            of the signal is determined by how many samples are recorded per 
            unit of time. For example, the illustration below shows an analog 
            pattern (represented as the curve) alongside a digital pattern 
            (represented as the discrete lines). 
            \cite{digitalSignalDefn}
        \addImage{H}{8cm}{digitalSignal.png}{Digital Signal \cite{digitalSignalImage}}{digital signal}
        \subsection{Discrete Signal}
        A discrete signal or discrete-time signal is a time series consisting 
        of a sequence of quantities. Unlike a continuous-time signal, a 
        discrete-time signal is not a function of a continuous argument; 
        however, it may have been obtained by sampling from a continuous-time 
        signal. When a discrete-time signal is obtained by sampling a sequence 
        at uniformly spaced times, it has an associated sampling rate.
        \cite{wikiDiscrete}
        \addImage{H}{7cm}{discreteSignal.png}{Discrete Signal \cite{wikiDiscrete}}{discrete signal}

    \section{Types of Discrete Time Signal}
        \subsection{Even and Odd Signal}
            \begin{enumerate}
                \item Even Signal
                \begin{enumerate}
                    \item[] A signal is said to be \textit{Even Signal} if it follows the following property: \newline $$x(-t) = x(t)$$
                \end{enumerate}
                \item Odd Signal
                \begin{enumerate}
                    \item[] A signal is said to be \textit{Odd Signal} if it follows the following property: \newline $$x(-t) = -x(t)$$
                \end{enumerate}
            \end{enumerate}
        \addImage{H}{8cm}{evenOdd.png}{Even-Odd signal Component synthesization\cite{evenOddImage}}{evenodd-signal}
        \subsection{Periodic and Aperiodic Signal}
            \begin{enumerate}
                \item Periodic Signal
                \item Aperiodic Signal
            \end{enumerate}

    \section{Energy and Power Signals}
        \begin{enumerate}
            \item Energy Signal\cite{energyPower}
                \begin{enumerate}
                    \item[] A signal which satisfies the following condition is  said to be energy signal.
                        $$ E = \int\limits_{-\infty}^{\infty} \left|x(t)\right|^2$$
                        Considering the energy in discrete time systems as:
                        $$ E = \sum\limits_{-\infty}^{\infty} \left|x[n]\right|^2$$
                \end{enumerate}
            \item Power Signal
                \begin{enumerate}
                    \item[] A signal which satisfies the following condition is  said to be power signal.
                        $$P = \lim_{T\to\infty}\frac{1}{2T} \int\limits_{-T}^{T}\left|x(t)^2\right|$$
                        Writing the power equation in discrete systems:
                        $$P = \lim_{N\to\infty}\frac{1}{2N+1} \sum\limits_{-T}^{T}\left|x[n]^2\right|$$
                \end{enumerate}
        \end{enumerate}

    \section{Block Diagram of DSP:}
        \addImage{H}{10cm}{dspBlockDiagram.jpg}{DSP block diagram\cite{blockDiagramDSP}}{dspBlockDiagram}
        \textbf{Description:} \newline
        \begin{itemize}
            \item ADC: an analog-to-digital converter is a system that 
            converts an analog signal, such as a sound picked up by a 
            microphone or light entering a digital camera, into a digital 
            signal. An ADC may also provide an isolated measurement such as an 
            electronic device that converts an input analog voltage or current 
            to a digital number representing the magnitude of the voltage or 
            current\cite{wikiADC}.
            \item DAC:  a digital-to-analog converter (DAC, D/A, D2A, or D-to-A)
            is a system that converts a digital signal into an analog signal. 
            An analog-to-digital converter (ADC) performs the reverse function.
            \cite{wikiDAC}
        \end{itemize}
    \section{Applications of DSP\cite(wikiAppDSP)}
        Applications of DSP include audio signal processing, audio compression,
        digital image processing, video compression, speech processing, speech 
        recognition, etc.
        \begin{enumerate}
            \item Speech Processing: Speech processing is the study of speech 
            signals and the processing methods of signals. The signals are 
            usually processed in a digital representation, so speech processing
            can be regarded as a special case of digital signal processing, 
            applied to speech signals. Aspects of speech processing includes 
            the acquisition, manipulation, storage, transfer and output of 
            speech signals. The input is called speech recognition and the 
            output is called speech synthesis. 
            \cite{wikiSpeechProcessing}
            \item Digital Image Processing: Digital Image Processing is the 
            use of computer algorithms to perform image processing on digital 
            images.[1] As a subcategory or field of digital signal processing, 
            digital image processing has many advantages over analog image 
            processing. It allows a much wider range of algorithms to be 
            applied to the input data and can avoid problems such as the 
            build-up of noise and signal distortion during processing. Since 
            images are defined over two dimensions (perhaps more) digital 
            image processing may be modeled in the form of multidimensional systems. 
            Digital image processing allows the use of much more complex 
            algorithms, and hence, can offer both more sophisticated 
            performance at simple tasks, and the implementation of methods 
            which would be impossible by analog means. \newline
            In particular, digital image processing is the only practical 
            technology for: \newline
            \begin{itemize}
                \item Classification
                \item Feature extraction
                \item Multi-scale signal analysis
                \item  Pattern recognition
                \item Projection
            \end{itemize}
        \end{enumerate}

    \section{Applications of IP\cite{appIP}}
        Some of the major fields in which digital image processing is widely 
        used are mentioned below
        \begin{itemize}
            \item Image sharpening and restoration
            \item Medical field
            \item Remote sensing
            \item Transmission and encoding
            \item Machine/Robot vision
            \item Color processing
            \item Pattern recognition
            \item Video processing
            \item Microscopic Imaging
        \end{itemize}
        \textbf{Image sharpening and restoration}
        \par
        Image sharpening and restoration refers here to process images that 
        have been captured from the modern camera to make them a better image 
        or to manipulate those images in way to achieve desired result. It 
        refers to do what Photoshop usually does. This includes Zooming, 
        blurring , sharpening , gray scale to color conversion, detecting 
        edges and vice versa , Image retrieval and Image recognition. The 
        common examples are: 
    \section{Advantages and Limitations of DSP}
        \subsection{Advantages of DSP}
        \subsection{Limitation of DSP}
    \section{Advantages and Limitations of IP} 
        \subsection{Advantages of IP} 
        \subsection{Disadvantages of IP} 
    \section{DSP vs Microprocessor}
    \end{multicols}

    \begin{thebibliography}{9}
    \bibitem{tpSignal} https://www.tutorialspoint.com/digital\_signal\_processing/dsp\_signals\_definition.htm
    \bibitem{wikiAnalog} https://en.wikipedia.org/wiki/Analog\_signal
    \bibitem{analogSignalImage} https://techdifferences.com/difference-between-analog-and-digital-signal.html
    \bibitem{digitalSignalDefn} https://www.chegg.com/homework-help/definitions/digital-signal-4
    \bibitem{digitalSignalImage} http://blogs.plymouth.ac.uk/embedded-systems/glossary-2/digital-signal-glossary-entry/
    \bibitem{wikiDiscrete} https://en.wikipedia.org/wiki/Discrete\_time\_and\_continuous\_time
    \bibitem{evenOddImage} http://www.songho.ca/dsp/signal/signals.html
    \bibitem{energyPower} https://www.chegg.com/homework-help/definitions/energy-and-power-in-signals-4
    \bibitem{blockdiagramDSP} https://slideplayer.com/slide/8916150/
    \bibitem{wikiADC} https://en.wikipedia.org/wiki/Analog-to-digital\_converter
    \bibitem{wikiDAC} https://en.wikipedia.org/wiki/Digital-to-analog\_converter
    \bibitem{wikiAppDSP} https://en.wikipedia.org/wiki/Digital\_signal\_processing
    \bibitem{wikiSpeechProcessing} https://en.wikipedia.org/wiki/Speech\_processing
    \bibitem{appIP} https://www.tutorialspoint.com/dip/applications\_and\_usage
    \end{thebibliography}

\end{document}
