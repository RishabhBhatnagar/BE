\documentclass{article}
\usepackage[margin=2cm]{geometry}
\usepackage{graphicx}
\usepackage{multicol}
\usepackage{url}

\graphicspath{./}
\setlength{\columnsep}{1cm}
\newenvironment{Figure}
 {\par\medskip\noindent{\linewidth}}
 {\endminipage\par\medskip}

\begin{document}
   \title{Case Study on DSIP and it's application.}                           
    \author{Rishabh Bhatnagar}
    \maketitle
    \tableofcontents
    \newpage

    \begin{thebibliography}{9}
    \bibitem{tpSignal} https://www.tutorialspoint.com/digital\_signal\_processing/dsp\_signals\_definition.htm
    \bibitem{wikiAnalog} https://en.wikipedia.org/wiki/Analog\_signal
    \end{thebibliography}
    
    \begin{multicols}{2}
    \section{Signal Definition}
    Anything that carries information can be called as signal. It can also be defined as a physical quantity that varies with time, temperature, pressure or with any independent variables such as speech signal or video signal. The process of operation in which the characteristics of a signal (Amplitude, shape, phase, frequency, etc.) undergoes a change is known as signal processing. 
    \cite{tpSignal}
    \section{Types of Signal}
        \subsection{Analog Signal}
        An analog signal is any continuous signal for which the time-varying feature (variable) of the signal is a representation of some other time varying quantity, i.e., analogous to another time varying signal. For example, in an analog audio signal, the instantaneous voltage of the signal varies continuously with the pressure of the sound waves.\cite{wikiAnalog}
        \begin{Figure}
         \centering
          \includegraphics[width=\linewidth]{analogSignals}
           \captionof{figure}{my caption of the figure}
           \end{Figure}
        \subsection{Digital Signal}
        \subsection{Discrete Signal}
    \section{Types of Discrete Time Signal}
        \subsection{Even and Odd Signal}
            \begin{enumerate}
                \item Even Signal
                \item Odd Signal
            \end{enumerate}
        \subsection{Periodic and Aperiodic Signal}
            \begin{enumerate}
                \item Periodic Signal
                \item Aperiodic Signal
            \end{enumerate}
    \section{Energy and Signal}
        \begin{enumerate}
            \item Energy Signal
            \item Power Signal
        \end{enumerate}
    \section{Block Diagram of DSP:}
    \section{Applications of DSP}
    \section{Applications of IP} 
    \section{Advantages and Limitations of DSP}
        \subsection{Advantages of DSP}
        \subsection{Limitation of DSP}
    \section{Advantages and Limitations of IP} 
        \subsection{Advantages of IP} 
        \subsection{Disadvantages of IP} 
    \section{DSP vs Microprocessor}
    \end{multicols}

\end{document}
