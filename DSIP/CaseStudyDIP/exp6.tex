\documentclass{article}


%%%%%%%%%%%%%%%%%%%%%%%%%%%%%Package Imports%%%%%%%%%%%%%%%%%%%%%%%%%%%%%%%%%%%
\usepackage[margin=2cm]{geometry}
\usepackage{graphicx}
\usepackage{multicol}
\usepackage{url}
\usepackage{float}
\usepackage{enumerate}

%%%%%%%%%%%%%%%%%%%%%%%%%%%%%Initial Settings%%%%%%%%%%%%%%%%%%%%%%%%%%%%%%%%%%
\graphicspath{./}
\setlength{\columnsep}{1cm}


%%%%%%%%%%%%%%%%%%%%%%%%%%%%%Custom Commands%%%%%%%%%%%%%%%%%%%%%%%%%%%%%%%%%%%
\newcommand{\addImage}[5]{
    \begin{figure}[#1]
        \centering
        \includegraphics[width=#2]{#3}
        \caption{#4}
        \label{fig:#5}
    \end{figure}
}

\begin{document}
   \title{Case Study on DSIP and it's application.}                           
    \author{Rishabh Bhatnagar}
    \maketitle
    \tableofcontents
    \newpage

    \begin{multicols}{2}                % two columns
		\section{Image}
			A digital image is a numeric representation, normally binary, of a 
			two-dimensional image. Depending on whether the image resolution is 
			fixed, it may be of vector or raster type. By itself, the term 
			"digital image" usually refers to raster images or bitmapped images 
			(as opposed to vector images). \cite{imageWiki}
			
			A digital image is a representation of a real image as a set of numbers that can be stored and handled by a digital computer. In order to translate the image into numbers, it is divided into small areas called pixels (picture elements). For each pixel, the imaging device records a number, or a small set of numbers, that describe some property of this pixel, such as its brightness (the intensity of the light) or its color. The numbers are arranged in an array of rows and columns that correspond to the vertical and horizontal positions of the pixels in the image. \cite{imageEncyclopedia} \\
			Digital images have several basic characteristics. One is the type of the image. For example, a black and white image records only the intensity of the light falling on the pixels. A color image can have three colors, normally RGB (Red, Green, Blue) or four colors, CMYK (Cyan, Magenta, Yellow, blacK). RGB images are usually used in computer monitors and scanners, while CMYK images are used in color printers. There are also non-optical images such as ultrasound or X-ray in which the intensity of sound or X-rays is recorded. In range images, the distance of the pixel from the observer is recorded. Resolution is expressed in the number of pixels per inch (ppi). A higher resolution gives a more detailed image. A computer monitor typically has a resolution of 100 ppi, while a printer has a resolution ranging from 300 ppi to more than 1440 ppi. This is why an image looks much better in print than on a monitor. \cite{imageEncyclopedia}
        %\addImage{H}{6cm}{analogSignal.jpeg}{Analog Signal}{analog signal}
        %\addImage{H}{8cm}{digitalSignal.png}{Digital Signal \cite{digitalSignalImage}}{digital signal}
        
        
        \section{Image Acquisition}
            In image processing, it is defined as the action of retrieving an image from some source, usually a hardware-based source for processing. It is the first step in the workflow sequence because, without an image, no processing is possible. The image that is acquired is completely unprocessed. \cite{buzztech}\\ 
            Now the incoming energy is transformed into a voltage by the combination of input electrical power and sensor material that is responsive to a particular type of energy being detected. The output voltage waveform is the response of the sensor(s) and a digital quantity is obtained from each sensor by digitizing its response. \cite{buzztech}
            \addImage{H}{6cm}{singleSensor.png}{Single Sensor}{single sensor}
            
            The general aim of Image Acquisition is to transform an optical image (Real World Data) into an array of numerical data which could be later manipulated on a computer, before any video orimage processing can commence an image must be captured by camera and converted into a manageable entity \cite{intext}. The Image Acquisition process consists of three steps:-
            \begin{enumerate}
                \item Optical system which focuses the energy
                \item Energy reflected from the object of interest
                \item A sensor which measure the amount of energy.
            \end{enumerate}

            Image Acquisition is achieved by suitable
            camera. We use different cameras for different
            application. If we need an x-ray image, we use a
            camera (film) that is sensitive to x-ray. If we want
            infra red image, we use camera which are sensitive
            to infrared radiation. For normal images (family
            pictures etc) we use cameras which are sensitive. \cite{acqPaper}
        
        \section{Image Processing\cite{sisu}}
            Image processing is a method to perform some operations on an image, in order to get an enhanced image or to extract some useful information from it. It is a type of signal processing in which input is an image and output may be image or characteristics/features associated with that image. Nowadays, image processing is among rapidly growing technologies. It forms core research area within engineering and computer science disciplines too.

        Image processing basically includes the following three steps:
        \begin{itemize}
            \item Importing the image via image acquisition tools;
            \item Analysing and manipulating the image;
            \item Output in which result can be altered image or report that is based on image analysis.
        \end{itemize}    

        \section{Types of Image Processing\cite{sisu}}
            There are two types of methods used for image processing namely, 
            \begin{itemize}
                \item Analogue Image Processing
                \item Digital Image Processing
            \end{itemize}
            \subsection{Digital Image Processing}
                Digital Image Processing is the use of computer algorithms to 
                perform image processing on digital images. As a subcategory or 
                field of digital signal processing, 
                digital image processing has many advantages over analog image 
                processing. It allows a much wider range of algorithms to be 
                applied to the input data and can avoid problems such as the 
                build-up of noise and signal distortion during processing. Since 
                images are defined over two dimensions (perhaps more) digital 
                image processing may be modeled in the form of multidimensional systems. 
                Digital image processing allows the use of much more complex 
                algorithms, and hence, can offer both more sophisticated 
                performance at simple tasks, and the implementation of methods 
                which would be impossible by analog means. \newline
                In particular, digital image processing is the only practical 
                technology for: \newline
                \begin{itemize}
                    \item Classification
                    \item Feature extraction
                    \item Multi-scale signal analysis
                    \item  Pattern recognition
                    \item Projection
                \end{itemize}
            \subsection{Analogue Image Processing}
                Analogue image processing can be used for the hard copies like printouts and photographs. Image analysts use various fundamentals of interpretation while using these visual techniques. Digital image processing techniques help in manipulation of the digital images by using computers. The three general phases that all types of data have to undergo while using digital technique are as follows:
                \begin{itemize}
                    \item pre-processing
                    \item enhancement
                    \item display, information extraction.
                \end{itemize}

        \section{Digital Image Processing system Block Diagram and Explanation}
        \addImage{H}{10cm}{DIPSteps.jpg}{Steps in DIP}{steps in DIP}
        \begin{enumerate}[(i)]
            \item \textbf{\underline{Image Acquisition :}} This is the first step or process of the fundamental steps of digital image processing. Image acquisition could be as simple as being given an image that is already in digital form. Generally, the image acquisition stage involves preprocessing, such as scaling etc.
            \item \textbf{\underline{Image Enhancement :}} Image enhancement is among the simplest and most appealing areas of digital image processing. Basically, the idea behind enhancement techniques is to bring out detail that is obscured, or simply to highlight certain features of interest in an image. Such as, changing brightness & contrast etc.
            \item \textbf{\underline{Image Restoration :}} Image restoration is an area that also deals with improving the appearance of an image. However, unlike enhancement, which is subjective, image restoration is objective, in the sense that restoration techniques tend to be based on mathematical or probabilistic models of image degradation.
            \item \textbf{\underline{Color Image Processing :}} Color image processing is an area that has been gaining its importance because of the significant increase in the use of digital images over the Internet. This may include color modeling and processing in a digital domain etc.
            \item \textbf{\underline{Wavelets and Multiresolution Processing :}} Wavelets are the foundation for representing images in various degrees of resolution. Images subdivision successively into smaller regions for data compression and for pyramidal representation.
            \item \textbf{\underline{Compression :}} Compression deals with techniques for reducing the storage required to save an image or the bandwidth to transmit it. Particularly in the uses of internet it is very much necessary to compress data.
            \item \textbf{\underline{Morphological Processing :}} Morphological processing deals with tools for extracting image components that are useful in the representation and description of shape.
            \item \textbf{\underline{Segmentation :}} Segmentation procedures partition an image into its constituent parts or objects. In general, autonomous segmentation is one of the most difficult tasks in digital image processing. A rugged segmentation procedure brings the process a long way toward successful solution of imaging problems that require objects to be identified individually.
            \item \textbf{\underline{Representation and Description :}} Representation and description almost always follow the output of a segmentation stage, which usually is raw pixel data, constituting either the boundary of a region or all the points in the region itself. Choosing a representation is only part of the solution for transforming raw data into a form suitable for subsequent computer processing. Description deals with extracting attributes that result in some quantitative information of interest or are basic for differentiating one class of objects from another.
            \item \textbf{\underline{Object recognition :}} Recognition is the process that assigns a label, such as,  “vehicle” to an object based on its descriptors.
            \item \textbf{\underline{Knowledge Base :}} Knowledge may be as simple as detailing regions of an image where the information of interest is known to be located, thus limiting the search that has to be conducted in seeking that information. The knowledge base also can be quite complex, such as an interrelated list of all major possible defects in a materials inspection problem or an image database containing high-resolution satellite images of a region in connection with change-detection applications.
        \end{enumerate}
        
        \section{Applications of IP}
            \begin{enumerate}[i]
                \item Image sharpening and restoration
                \item Medical field
                \item Remote sensing
                \item Transmission and encoding
                \item Machine/Robot vision
                \item Color processing
                \item Pattern recognition
                \item Video processing
                \item Microscopic Imaging
            \end{enumerate}
        \section{Application of IP in Robotics}
            \subsection{Hurdle detection}
                Hurdle detection is one of the common task that has been done through image processing, by identifying different type of objects in the image and then calculating the distance between robot and hurdles.
                \addImage{H}{8cm}{hurdle.jpg}{Hurdle Detection}{Hurdle Detection}
            \subsection{Line Follower Robot}
                Most of the robots today work by following the line and thus are called line follower robots. This help a robot to move on its path and perform some tasks. This has also been achieved through image processing.
                \addImage{H}{8cm}{lineFollower.jpg}{Line Follower}{line Follower}
        
        
        \section{Explain Different Image file format}
            \begin{itemize} 
                \item \textbf{BMP:} Short for "Bitmap" .It can be pronounced as "bump," "B-M-P," or simply a "bitmap image." The BMP format is a commonly used raster graphic format for saving image files. It was introduced on the Windows platform, but is now recognized by many programs on both Macs and PCs.The BMP format stores color data for each pixel in the image without any compression. For example, a 10x10 pixel BMP image will include color data for 100 pixels. This method of storing image information allows for crisp, high-quality graphics, but also produces large file sizes. The JPEG and GIF formats are also bitmaps, but use image compression algorithms that can significantly decrease their file size. For this reason, JPEG and GIF images are used on the Web, while BMP images are often used for printable images.

                \item \textbf{TIFF}: Also known as TIF, file types ending in .tif. TIFF stands for Tagged Image File Format. TIFF images create very large file sizes. TIFF images are uncompressed and thus contain a lot of detailed image data (which is why the files are so big) TIFFs are also extremely flexible in terms of color (they can be grayscale, or CMYK for print, or RGB for web) and content (layers, image tags).TIFF is the most common file type used in photo software (such as Photoshop), as well as page layout software (such as Quark and InDesign), again because a TIFF contains a lot of image data.

                \item \textbf{JPEG}: Also known as JPG, file types ending in .jpg. JPEG stands for Joint Photographic Experts Group, which created this standard for this type of image formatting. JPEG files are images that have been compressed to store a lot of information in a small-size file. Most digital cameras store photos in JPEG format, because then you can take more photos on one camera card than you can with other formats. A JPEG is compressed in a way that loses some of the image detail during the compression in order to make the file small (and thus called “lossy” compression).JPEG files are usually used for photographs on the web, because they create a small file that is easily loaded on a web page and also looks good.JPEG files are bad for line drawings or logos or graphics, as the compression makes them look “bitmappy” (jagged lines instead of straight ones).
            \end{itemize}
    \begin{thebibliography}{7}
	\bibitem{imageEncyclopedia} https://www.encyclopedia.com/computing/news-wires-white-papers-and-books/digital-images
	\bibitem{imageWiki} https://www.encyclopedia.com/computing/news-wires-white-papers-and-books/digital-images
	\bibitem{buzztech} https://buzztech.in/image-acquisition-in-digital-image-processing/
	\bibtem{buzztechImage} https://buzztech.in/wp-content/uploads/2017/12/Screen-Shot-2017-12-16-at-8.18.49-AM.png
    \bibitem{intext} D. Sugimura, T. Mikami, H. Yamashita, and T.Hamamoto, “Enhancing Color Images of Extremely Low Light Scenes Based on RGB/NIR Images Acquisition With Different Exposure Times”, IEEE TRANSACTIONS ON IMAGE PROCESSING, VOL. 24, NO. 11, NOVEMBER 2015.
    \bibitem{acqPaper} https://www.researchgate.net/publication/318500799-Image-Acquisition-and-Techniques-to-Perform-Image-Acquisition
    \bibitem{sisu} https://sisu.ut.ee/imageprocessing/book/1
    
    \end{thebibliography}
    \end{multicols}
\end{document}
