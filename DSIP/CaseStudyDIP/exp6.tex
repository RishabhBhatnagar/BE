\documentclass{article}


%%%%%%%%%%%%%%%%%%%%%%%%%%%%%Package Imports%%%%%%%%%%%%%%%%%%%%%%%%%%%%%%%%%%%
\usepackage[margin=2cm]{geometry}
\usepackage{graphicx}
\usepackage{multicol}
\usepackage{url}
\usepackage{float}

%%%%%%%%%%%%%%%%%%%%%%%%%%%%%Initial Settings%%%%%%%%%%%%%%%%%%%%%%%%%%%%%%%%%%
\graphicspath{./}
\setlength{\columnsep}{1cm}


%%%%%%%%%%%%%%%%%%%%%%%%%%%%%Custom Commands%%%%%%%%%%%%%%%%%%%%%%%%%%%%%%%%%%%
\newcommand{\addImage}[5]{
    \begin{figure}[#1]
        \centering
        \includegraphics[width=#2]{#3}
        \caption{#4}
        \label{fig:#5}
    \end{figure}
}

\begin{document}
   \title{Case Study on DSIP and it's application.}                           
    \author{Rishabh Bhatnagar}
    \maketitle
    \tableofcontents
    \newpage

    \begin{multicols}{2}                % two columns
		\section{Image}
			A digital image is a numeric representation, normally binary, of a 
			two-dimensional image. Depending on whether the image resolution is 
			fixed, it may be of vector or raster type. By itself, the term 
			"digital image" usually refers to raster images or bitmapped images 
			(as opposed to vector images). \cite{imageWiki}
			
			A digital image is a representation of a real image as a set of numbers that can be stored and handled by a digital computer. In order to translate the image into numbers, it is divided into small areas called pixels (picture elements). For each pixel, the imaging device records a number, or a small set of numbers, that describe some property of this pixel, such as its brightness (the intensity of the light) or its color. The numbers are arranged in an array of rows and columns that correspond to the vertical and horizontal positions of the pixels in the image. \cite{imageEncyclopedia} \\
			Digital images have several basic characteristics. One is the type of the image. For example, a black and white image records only the intensity of the light falling on the pixels. A color image can have three colors, normally RGB (Red, Green, Blue) or four colors, CMYK (Cyan, Magenta, Yellow, blacK). RGB images are usually used in computer monitors and scanners, while CMYK images are used in color printers. There are also non-optical images such as ultrasound or X-ray in which the intensity of sound or X-rays is recorded. In range images, the distance of the pixel from the observer is recorded. Resolution is expressed in the number of pixels per inch (ppi). A higher resolution gives a more detailed image. A computer monitor typically has a resolution of 100 ppi, while a printer has a resolution ranging from 300 ppi to more than 1440 ppi. This is why an image looks much better in print than on a monitor. \cite{imageEncyclopedia}
        %\addImage{H}{6cm}{analogSignal.jpeg}{Analog Signal}{analog signal}
        %\addImage{H}{8cm}{digitalSignal.png}{Digital Signal \cite{digitalSignalImage}}{digital signal}
        
        
        \section{Image Acquisition}
            In image processing, it is defined as the action of retrieving an image from some source, usually a hardware-based source for processing. It is the first step in the workflow sequence because, without an image, no processing is possible. The image that is acquired is completely unprocessed. \cite{buzztech}\\ 
            Now the incoming energy is transformed into a voltage by the combination of input electrical power and sensor material that is responsive to a particular type of energy being detected. The output voltage waveform is the response of the sensor(s) and a digital quantity is obtained from each sensor by digitizing its response. \cite{buzztech}
            \addImage{H}{6cm}{singleSensor.png}{Single Sensor}{single sensor}
            
            The general aim of Image Acquisition is to transform an optical image (Real World Data) into an array of numerical data which could be later manipulated on a computer, before any video orimage processing can commence an image must be captured by camera and converted into a manageable entity \cite{intext}. The Image Acquisition process consists of three steps:-
            \begin{enumerate}
                \item Optical system which focuses the energy
                \item Energy reflected from the object of interest
                \item A sensor which measure the amount of energy.
            \end{enumerate}

            Image Acquisition is achieved by suitable
            camera. We use different cameras for different
            application. If we need an x-ray image, we use a
            camera (film) that is sensitive to x-ray. If we want
            infra red image, we use camera which are sensitive
            to infrared radiation. For normal images (family
            pictures etc) we use cameras which are sensitive. \cite{acqPaper}
        
        \section{Image Processing\cite{sisu}}
            Image processing is a method to perform some operations on an image, in order to get an enhanced image or to extract some useful information from it. It is a type of signal processing in which input is an image and output may be image or characteristics/features associated with that image. Nowadays, image processing is among rapidly growing technologies. It forms core research area within engineering and computer science disciplines too.

        Image processing basically includes the following three steps:
        \begin{itemize}
            \item Importing the image via image acquisition tools;
            \item Analysing and manipulating the image;
            \item Output in which result can be altered image or report that is based on image analysis.
        \end{itemize}    

        \section{Types of Image Processing\cite{sisu}}
            There are two types of methods used for image processing namely, 
            \begin{itemize}
                \item Analogue Image Processing
                \item Digital Image Processing
            \end{itemize}
            \subsection{Digital Image Processing}
                Digital Image Processing is the use of computer algorithms to 
                perform image processing on digital images. As a subcategory or 
                field of digital signal processing, 
                digital image processing has many advantages over analog image 
                processing. It allows a much wider range of algorithms to be 
                applied to the input data and can avoid problems such as the 
                build-up of noise and signal distortion during processing. Since 
                images are defined over two dimensions (perhaps more) digital 
                image processing may be modeled in the form of multidimensional systems. 
                Digital image processing allows the use of much more complex 
                algorithms, and hence, can offer both more sophisticated 
                performance at simple tasks, and the implementation of methods 
                which would be impossible by analog means. \newline
                In particular, digital image processing is the only practical 
                technology for: \newline
                \begin{itemize}
                    \item Classification
                    \item Feature extraction
                    \item Multi-scale signal analysis
                    \item  Pattern recognition
                    \item Projection
                \end{itemize}
            \subsection{Analogue Image Processing}
                Analogue image processing can be used for the hard copies like printouts and photographs. Image analysts use various fundamentals of interpretation while using these visual techniques. Digital image processing techniques help in manipulation of the digital images by using computers. The three general phases that all types of data have to undergo while using digital technique are as follows:
                \begin{itemize}
                    \item pre-processing
                    \item enhancement
                    \item display, information extraction.
                \end{itemize}

        \section{Digital Image Processing system Block Diagram and Explanation}
        \section{Applications of IP}
        \section{Explain any 1 Application in Detail}
        \section{Explain Different Image file format}










        \begin{enumerate}
                \item Even Signal
                \begin{enumerate}
                    \item[] A signal is said to be \textit{Even Signal} if it follows the following property: \newline $$x(-t) = x(t)$$
                \end{enumerate}
                \item Odd Signal
                \begin{enumerate}
                    \item[] A signal is said to be \textit{Odd Signal} if it follows the following property: \newline $$x(-t) = -x(t)$$
                \end{enumerate}
            \end{enumerate}
        
    \end{multicols}

    \section{DSP vs Microprocessor}
        \begin{tabular}{ ||p{1cm}|p{4cm}|p{6cm}|p{6cm}||}
            \hline
            \textbf{Sr. No.}&\textbf{Parameters}&\textbf{DSP Processor}&\textbf{MicroProcessor}\\
            \hline
            1&Instruction cycle&Instructions are executed in single cycle of the clock&Multiple clocks cycles are required for execution of one instruction.\\\hline
            2&Instruction execution&Parallel execution is possible.&Execution of instruction is always sequential.\\\hline
            3&Memories&Separate data and program memory.&No such separate memories are present.\\\hline
            4&On chip/Off chip memories&Program and Data memories are present on chip extendable off chip.&Normally on chip cache memory present, main memory is off chip.\\\hline
            5&Program flow control&Program sequencer and instruction register take care of program flow&Program counter take care of flow of execution.\\\hline
            6&Pipelining&Pipelining is implicate through instruction register and instruction cache.&Queuing is perform explicate by one queue register to support pipelining.\\
            \hline
        \end{tabular}

    \begin{thebibliography}{9}
	\bibitem{imageEncyclopedia} https://www.encyclopedia.com/computing/news-wires-white-papers-and-books/digital-images
	\bibitem{imageWiki} https://www.encyclopedia.com/computing/news-wires-white-papers-and-books/digital-images
	\bibitem{buzztech} https://buzztech.in/image-acquisition-in-digital-image-processing/
	    \bibtem{buzztechImage} https://buzztech.in/wp-content/uploads/2017/12/Screen-Shot-2017-12-16-at-8.18.49-AM.png
    \bibitem{intext} D. Sugimura, T. Mikami, H. Yamashita, and T.Hamamoto, “Enhancing Color Images of Extremely Low Light Scenes Based on RGB/NIR Images Acquisition With Different Exposure Times”, IEEE TRANSACTIONS ON IMAGE PROCESSING, VOL. 24, NO. 11, NOVEMBER 2015.
    \bibitem{acqPaper} https://www.researchgate.net/publication/318500799_Image_Acquisition_and_Techniques_to_Perform_Image_Acquisition
    \bibitem{sisu} https://sisu.ut.ee/imageprocessing/book/1
    
    \end{thebibliography}

\end{document}
